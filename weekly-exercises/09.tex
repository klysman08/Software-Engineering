% Created 2019-05-13 Mon 14:44
% Intended LaTeX compiler: pdflatex
\documentclass[11pt]{article}
\usepackage[utf8]{inputenc}
\usepackage[T1]{fontenc}
\usepackage{graphicx}
\usepackage{grffile}
\usepackage{longtable}
\usepackage{wrapfig}
\usepackage{rotating}
\usepackage[normalem]{ulem}
\usepackage{amsmath}
\usepackage{textcomp}
\usepackage{amssymb}
\usepackage{capt-of}
\usepackage{hyperref}
\usepackage[utf8]{inputenc}
\usepackage{enumerate}
\usepackage[a4paper, margin=2cm]{geometry}
\usepackage{indentfirst}
\usepackage[, brazilian]{babel}
\usepackage{float}
\usepackage{color, colortbl}
\usepackage{titling}
\usepackage[labelformat=empty]{caption}
\setlength{\droptitle}{-1.5cm}
\hypersetup{ colorlinks = true, urlcolor = blue }
\definecolor{beige}{rgb}{0.93,0.93,0.82}
\definecolor{brown}{rgb}{0.4,0.2,0.0}
\documentclass{article} % What kind of document this is
\author{Fernanda Guimarães - 2016058166}
\date{}
\title{Exercício 9 - (entrega 13/05)}
\hypersetup{
 pdfauthor={Fernanda Guimarães - 2016058166},
 pdftitle={Exercício 9 - (entrega 13/05)},
 pdfkeywords={},
 pdfsubject={},
 pdfcreator={Emacs 26.2 (Org mode 9.2)},
 pdflang={Brazilian}}
\begin{document}

\maketitle

\section*{Questão1}
\label{sec:org989dc8e}
\begin{verbatim}
class Fibonacci {
  private int first_number;
  private int second_number;

  public Fibonacci (int first_number, int second_number) {
    this.first_number = first_number;
    this.second_number = second_number;
  }


  public int next(){
    int next = this.first_number + this.second_number;
    return next;
  }

  public static void main(String[] args) {
    int a1 = 1;
    int a2 = 2;

    Fibonacci fibonacci = new Fibonacci(a1, a2);
    System.out.println(fibonacci.next());
  }
}

\end{verbatim}

\begin{verbatim}
import org.junit.jupiter.api.Test;
import static org.junit.Assert.assertTrue;


@Test
class FibonacciTest {
  public void testNext() throws Exception {
    Fibonacci fibonacci = new Fibonacci(0,1);
    int result = fibonacci.next();
    int expected = 1;

    assertTrue(expected.equals(result));
  }
}
\end{verbatim}
\section*{Questão 2}
\label{sec:org410dcd3}
\begin{verbatim}
@Test
public void testFooThrowsIndexOutOfBoundsException() {
    assertThrows(IndexOutOfBoundsException.class, () -> foo.doStuff(),
           "Expected foo() to throw, but it didn't");
}
\end{verbatim}
\end{document}
