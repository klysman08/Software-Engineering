% Created 2019-03-17 Sun 20:19
% Intended LaTeX compiler: pdflatex
\documentclass[11pt]{article}
\usepackage[utf8]{inputenc}
\usepackage[T1]{fontenc}
\usepackage{graphicx}
\usepackage{grffile}
\usepackage{longtable}
\usepackage{wrapfig}
\usepackage{rotating}
\usepackage[normalem]{ulem}
\usepackage{amsmath}
\usepackage{textcomp}
\usepackage{amssymb}
\usepackage{capt-of}
\usepackage{hyperref}
\usepackage{minted}
\usepackage[a4paper, margin=2cm]{geometry}
\usepackage{indentfirst}
\usepackage[, brazilian]{babel}
\usepackage{float}
\usepackage{color, colortbl}
\usepackage{titling}
\setlength{\droptitle}{-1.5cm}
\hypersetup{ colorlinks = true, urlcolor = blue }
\usemintedstyle{murphy}
\definecolor{beige}{rgb}{0.93,0.93,0.82}
\definecolor{brown}{rgb}{0.4,0.2,0.0}
\author{Fernanda Guimarães - 2016058166}
\date{}
\title{Exercício 02 - (entrega 18/03)}
\hypersetup{
 pdfauthor={Fernanda Guimarães - 2016058166},
 pdftitle={Exercício 02 - (entrega 18/03)},
 pdfkeywords={},
 pdfsubject={},
 pdfcreator={Emacs 26.1 (Org mode 9.2)}, 
 pdflang={Brazilian}}
\begin{document}

\maketitle

\section{Questão 1}
\label{sec:orgc97a2cf}
Testes A/B são uma prática experimental para descobrir o que os usuários querem, em
detrimento de tentar levantar requisitos avançadamente. A prática de desenvolvimento que
permite "live experimentation" usando tais testes é o \emph{continuos deployment}: as
inovações dos engenheiros são implantadas imediatamente para usuários reais
experimentarem. Isso permite que engenheiros aprendam sobre a diversidade de usuários, e
aprenciem suas diferentes visões sobre o Facebook.

Para melhoras os dados obtidos de testes, o Facebook se utiliza de testes de usabilidade
com grupos de foco de usuário em adição a testes do produto implantado em larga escala.

\section{Questão 2}
\label{sec:org45ef749}
Um terço dos arquivos foram editados por apenas um engenheiro, e outro quarto por dois.
Apenas 10\% dos arquivos são lidados por mais de sete engenheiros. Isso ocorre por dois
motivos: 
\begin{itemize}
\item No Facebook, engenheiros conduzem qualquer tipo de testes de unidade para o seu
código. Em adição, o código tem que passar em todos os testes acumulados de regressão,
administrados automaticamente como parte do processo de commit e push.
\item Desenvolvedores também têm que dar suporte ao uso operacional do seu software — uma
combinação que ficou conhecida como “devops.” As metodologias e ferramentas não são
suficientes por si mesmas porque elas podem ser sempre usadas
erroneamente. Conseguintemente, uma cultura de responsabilidade pessoal é crítica.
\end{itemize}

\section{Questão 3}
\label{sec:org8d34938}
O desenvolvimento de software no Facebook é contrário a muitas práticas comuns da
indústria. Os motivos são diversos, dentre eles:
\begin{itemize}
\item Não há um plano detalhado para atingir um produto final, bem especificado.
\item Engenheiros trabalham diretamente com uma base de código sem branches e sem merging.
\item Não há um time separado de QA para testes.
\item Código novo é liberado em alto ritmo, atualmente duas vezes por dia.
\item Engenheiros têm que escolher por si em que vão trabalhar.
\item Não há punição por falhas.
\end{itemize}

\section{Questão 4}
\label{sec:orgddd300a}
\begin{itemize}
\item Post de dúvidas: o fórum, serve, primeiramente, para a postagem de dúvidas a serem
respondidas por usuários, com comentários e avaliação para as melhores respostas. No
fim, o autor da pergunta escolha a resposta que mais se adequou à pergunta.
\item Respostas: o fórum também serve para pessoas com alto conhecimento em uma ou mais
áreas acumularem pontos com respostas e comentários. Esses pontos podem usados para
futuras contratações, satisfação pessoal ou puramente altruísmo.
\item Pesquisa: o fórum também serve para usuários pesquisarem por perguntas já respondidas
previamente. Com o passar dos anos, apenas perguntas muito específicas, ou de
ferramentas novas, serão novidade.
\item Aprendizado de Máquina: fóruns de Q\&A podem ser usados em competições de machine
learning para a tentativa de previsão de tags automáticas para as perguntas.
\item Estudos de confiabilidade: muitos exemplos de código de fóruns online de Q\&A podem não
ser confiáveis. Esse fato pode ser usado em estudos empíricos para uma análise de mal
uso de API.
\end{itemize}

\section{Questão 5}
\label{sec:orgcc34e0b}
Eu tentaria convencê-lo, pois de fato o usuário não conseguiria utilizar essa
funcionalidade sem ter aberto o arquivo primeiro.
\end{document}
