% Created 2019-04-02 Tue 19:56
% Intended LaTeX compiler: pdflatex
\documentclass[11pt]{article}
\usepackage[utf8]{inputenc}
\usepackage[T1]{fontenc}
\usepackage{graphicx}
\usepackage{grffile}
\usepackage{longtable}
\usepackage{wrapfig}
\usepackage{rotating}
\usepackage[normalem]{ulem}
\usepackage{amsmath}
\usepackage{textcomp}
\usepackage{amssymb}
\usepackage{capt-of}
\usepackage{hyperref}
\usepackage{minted}
\usepackage[a4paper, margin=2cm]{geometry}
\usepackage{indentfirst}
\usepackage[, brazilian]{babel}
\usepackage{float}
\usepackage{color, colortbl}
\usepackage{titling}
\setlength{\droptitle}{-1.5cm}
\hypersetup{ colorlinks = true, urlcolor = blue }
\usemintedstyle{murphy}
\definecolor{beige}{rgb}{0.93,0.93,0.82}
\definecolor{brown}{rgb}{0.4,0.2,0.0}
\author{Fernanda Guimarães - 2016058166}
\date{}
\title{Resumo prova 01}
\hypersetup{
 pdfauthor={Fernanda Guimarães - 2016058166},
 pdftitle={Resumo prova 01},
 pdfkeywords={},
 pdfsubject={},
 pdfcreator={Emacs 26.1 (Org mode 9.2)}, 
 pdflang={Brazilian}}
\begin{document}

\maketitle
DCC/UFMG - Engenharia de Software - Prof. Marco Tulio Valente (mtov@dcc.ufmg.br)

Revisão para a Prova 1:

\section{Qual a diferença entre requisitos funcionais e não-funcionais?}
\label{sec:org21db323}
\begin{itemize}
\item Funcionais: "o que" um sistema deve fazer; quais funcionalidades ou serviços ele deve
implementar. Exemplo: quando uma nota de uma prova for lançada, os alunos devem ser
notificados por e-mail
\item Não-funcionais: "como" um sistema deve operar, sob quais "constraints" e qual
qualidade de serviço ele deve oferecer. Exemplo: o e-mail acima deve chegar em até um
minuto.
\end{itemize}
\section{Descreva cinco tipos de requisitos não-funcionais.}
\label{sec:org2fd4326}
\begin{itemize}
\item Desempenho: "deve dar o saldo da conta em menos de 5 segundos"
\item Disponibilidade: "deve estar no ar 99.99\% do tempo"
\item Capacidade: "deve ser capaz de armazenar dados de 1M de clientes"
\item Tolerância a falhas: "deve continuar operando mesmo se São Paulo cair"
\item Segurança: "deve criptografar todos os dados trocados com as agências"
\end{itemize}

\section{Descreva o principal benefício de um bom projeto modular:}
\label{sec:org97abdd6}
\begin{itemize}
\item Paraleliza o desenvolvimento (cada "time" trabalha em um módulo)
\item Facilita o entendimento (você pode começar a contribuir após dominar um único módulo)
\item Facilita manutenções (que tendem a ficar "isoladas" em módulo)
\end{itemize}

\section{Todo defeito ou bug em um software causa uma falha? Sim ou Não? Justifique.}
\label{sec:org7a30b57}
Nem todo defeito/bug causa falhas; pois o código defeituoso pode nunca vir a ser executado/testado.
\section{Defina refactoring. Dê três exemplos de refactoring.}
\label{sec:org01c5098}
Refactoring: manutenção (perfectiva) realizada exclusivamente para incrementar
manutenibilidade; ou seja, não corrige bug, não implementa nova funcionalidade etc

Exemplos clássicos:
\begin{itemize}
\item Rename variable/class/etc
\item Extract function/class/interface/package
\item Move function/class
\end{itemize}
\section{Defina monorepos.}
\label{sec:orgb23b962}
Um único repositório monolítico (isto é, um monorepo), gerenciado por um sistema de
controle de versão proprietário

\section{Essencialmente, qual a principal motivação (ou inspiração) de processos de desenvolvimento do tipo Waterfall. Isto é, por que eles foram os processos propostos inicialmente para desenvolvimento de software?}
\label{sec:org9f4732f}
Muito usado quando os custos de uma falha de design ou implementação podem ser enormes,
até em termos de vidas humanao. Parte do sucesso pode ser explicada pela sua
"padronização" pelo Departamento de Defesa Norte-Americano, em 1985.

\section{Liste três fatores de qualidade externa de software. Liste três fatores de qualidade interna.}
\label{sec:org166b467}
Qualidade Externa: Correção, robustez, reusabilidade.
Qualidade Interna: Modularidade, legibilidade, testabilidade.  

\section{Explique e descreva a classificação de sistemas de software do tipos A, B, C.}
\label{sec:org6e918fd}
Sistemas C (casuais): 
\begin{itemize}
\item não existe pressão para níveis altos de qualidade
\item podem ter bugs
\item 1 ou 2 engenheiros
\item sistemas pequenos, algo sem importância
\item tipo mais comum
\end{itemize}

Sistemas B (business):
\begin{itemize}
\item sistemas críticos (geram lucro, etc)
\item risco: não usarem técnicas de ES e se tornarem um "passivo" ao invés de um "ativo" para a empresa
\end{itemize}

Sistemas A (acute):
\begin{itemize}
\item sistemas onde nada pode dar errado, pois o custo é imenso, em termos de vidas humanas
e/ou grandes montantes financeiros
\item exemplos: sistemas de transporte, médicos, aviação, espaciais etc
\item requerem certificações normalmente
\end{itemize}


\section{Qual a principal diferença entre XP e Scrum? Qual a principal diferença entre Kanban e Scrum? Qual a principal diferença entre Lean e Scrum?}
\label{sec:org59e942b}
Diferença XP e Scrum:
\begin{itemize}
\item Scrum é um método ágil para gerenciamento de projetos, mas não necessariamente de
software
\item Scrum não tem preocupação com práticas, pois seu objetivo é mais amplo.
\end{itemize}
Diferença Kanban e Scrum:
Kanbam e Scrum:
\begin{itemize}
\item Scrum - sprints + "quota" de tarefas em cada estágio
\item Também não tem scrum master, product owner, pelo menos explicitamente
\end{itemize}
Lean e Scrum:
\begin{itemize}
\item Lean é um método "ágil" para gerenciamento de startups, com ciclos curtos e feedback,
ou seja, é mais específico que o Scrum.
\end{itemize}

\section{Descreva um valor de XP? Descreva dois princípios de XP? Descreva três práticas de desenvolvimento propostas por XP?}
\label{sec:org5c6e362}
Valor: feedback constante: pois em sistemas de software, dada à complexidade dos mesmos,
  pode ser difícil ter a solução "certa", logo de início; 

Princípios:
\begin{itemize}
\item Flow: priorizar um fluxo contínuo e produtivo de atividades.
\item Baby Steps: uma feature de cada vez (a de maior prioridade para o cliente); uma sub-feature de cada
\end{itemize}
vez etc.

Práticas:
\begin{itemize}
\item Pair programming: toda tarefa de programação (design, codificação, testes) deve ser
feita por dois programadores, em conjunto;
\item Iteração: planeje seu trabalho semanalmente, de forma que ao fim de toda semana
deve-se produzir "deployable software".
\item TDD: não só ter muitos testes, mas escrever esses testes antes da fase de codificação
(e claro, são sempre testes automatizados).
\end{itemize}



\section{O que é uma slack no contexto de XP?}
\label{sec:org828947e}
Introduza nos ciclos algumas "folgas", isto é, tarefas menos importantes, que possam ser descartadas, caso o projeto fique atrasado
\begin{itemize}
\item Refatorações
\item Pesquisa e prospecção de novas tecnologias
\item Alguma documentação ou manual de uso
\item Seminários internos etc
\end{itemize}

\section{Basicamente, quando ocorre um conflito de integração?}
\label{sec:org97ba939}
\begin{itemize}
\item Ao implementar F, você alterou um arquivo X
\item Ao mesmo tempo, outro desenvolvedor alterou as mesmas linhas de X e integrou o código dele! Antes de você integrar ….
\item O que ocorre quando você for integrar o seu código? Um conflito
\end{itemize}

\section{Por que XP advoga integrações contínuas?}
\label{sec:org2a830a4}
(1) feedback rápido; (2) evitar "integration-hell" (isto é, quando a integração é mais
custosa que o desenvolvimento)

\section{Qual a diferença entre integração contínua, continuous delivery e continuous deployment?}
\label{sec:orge1498b9}
\begin{itemize}
\item CI: continuous integration - prática na qual membros do time integram seus trabalhos
frequentemente. Testes automáticos. Everyone merging code changes to a central
repository multiple times a day.
\item CY: continuous delivery: do CI, plus automatically prepare and track a release to
production (anyone with sufficient privileges can do this in a few clicks).
\item CD continuous deployment: é uma evolução de Continuous Integration (CI)- logo após
integrado, código é também liberado para uso. Continuous deployment is like continuous
delivery, except that releases happen automatically.
\end{itemize}

\section{Cite duas práticas propostas por XP que acabaram não sendo largamente adotadas.}
\label{sec:org4114a37}
Story points e pair programming.

\section{Em projetos XP, não existe a figura de um arquiteto de software, já que o projeto/arquitetura não são definidos up front (isto é, eles também são incrementais). Verdadeiro ou Falso. Justifique.}
\label{sec:orgc5109fb}
Falso, podem existir arquitetos. A arquitetura também é definida de forma incremental; o
sistema vai sendo particionado, à medida que evolui


\section{Quem define o tamanho das estórias em times XP? O que é a velocidade de um time?}
\label{sec:org8581110}
São estimadas pelos programadores; que definem quanto tempo leva-se para implementar
cada estória (porém, não podem ser muito complexas). 

Velocidade: capacidade de trabalho de um time, em uma iteração Exemplo: a velocidade de
um time = 26 (significa que ele é capaz de implementar 26 story points em uma iteração)


\section{No contexto de Scrum, o que é grooming?}
\label{sec:orgae94e98}
Tarefas de "cuidar" do product backlog, incluindo: (1) criar e refinar estórias; (2)
estimar estórias; (3) priorizar estórias. O time todo realiza, chefiado pelo product
owner.

\section{Qual a diferença entre sprint review e sprint retrospectiva, em Scrum.}
\label{sec:org731c98b}
Sprint Review:
\begin{itemize}
\item Reunião que marca o fim de um sprint, com participação inclusive de stakeholders
envolvidos com o resultado do sprint
\end{itemize}
Retrospectiva:
\begin{itemize}
\item Reunião interna para refletir sobre o processo e melhorá-lo. Última atividade do
sprint.
\end{itemize}

\section{Suponha que você está fazendo um sistema para um público externo e amplo (por exemplo, um sistema de Q\&A, como o do trabalho), usando Scrum. Como escolher o product owner neste caso?}
\label{sec:org2280250}
\begin{itemize}
\item Product Owner pode ser um cliente externo (por exemplo, quando foi o cliente externo que
contratou o desenvolvimento do sistema).
\item Ou Product Owner pode ser alguém da área de marketing ou Vendas da organização que
está desenvolvendo o software (isto é, alguém que vai representar os clientes reais).
\end{itemize}

\section{Por que cada atividade de um quadro Kanban possui duas colunas? Qual o significado destas duas colunas?}
\label{sec:org4ed07fd}
Com exceção do Backlog, demais fases possuem duas colunas:
\begin{itemize}
\item 1a coluna: items em andamento na fase (ex.: em implementação)
\item 2a coluna: itens que já terminaram essa fase, mas que
ainda não foram movidos para a fase seguinte (ex.: implementados e aguardando entrarem
em validação; é uma espécie de buffer para a fase seguinte)
\end{itemize}

\section{Essencialmente, por que o modelo em Espiral não é classificado como um modelo de desenvolvimento ágil?}
\label{sec:orgd7afb1a}
Most Spiral models of development still insist on big, up-front design. The emphasis is
on knowing as much as you can about how the system will be used; discovering all the use
cases. Once you know these, then you design the system and break it down into phases that
follow an iterative detail-design, implementation, test, refactor-design loop.
\section{Quais as fases propostas pelo Método RUP?}
\label{sec:orgcd95272}
\begin{itemize}
\item Inception: análise de viabilidade, orçamentos e definição de escopo
\item Elaboração: especificação de requisitos (via casos de uso) e da arquitetura
\item Construção: projeto de mais baixo nível, implementação e testes
\item Transição: disponibilização do sistema para produção
\item Essas fases podem ser divididas em iterações (ex.: construção pode ter 4 iterações); pode-se também iterar sobre as 4 fases (como em Espiral)
\end{itemize}


\section{Quando não usar métodos ágeis?}
\label{sec:org98dc649}
\begin{itemize}
\item Requisitos estáveis
\item Design conhecido e simples
\item Desenvolvedores dominam a área do projeto
\item Baixo risco
\item Custos de mudanças é alto
\end{itemize}
\end{document}
